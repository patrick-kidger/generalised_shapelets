\section{Experiments}
Our generalised shapelet transform, contrasted with the classical shapelet transform, has two extra degrees of freedom: the choice of interpolation scheme $\iota$, and the choice of discrepancy function $\pi^A_S$. In our experiments, we consider $\pi^A_S$ given by either of equations \eqref{eq:learnt-discrepancy} or \eqref{eq:logsignature-discrepancy}. Meanwhile, we take $\iota$ to be piecewise linear interpolation, because efficient algorithms for computing the logsignature transform only exist for piecewise linear paths \cite{signatory}.

TODO: we need to consider more disrepancy functions than this.


\subsection{The UEA (Multivariate) Time-Series Archive}
We begin by comparing the classification performance of the old method of shapelets to our new `generalised' approach for which we consider three learnt metrics: a standard L2-metric, an learnt metric and a logsig-3-diagonal. We evaluate the methods on a subset of the UEA time-series archive \cite{bagnall2018uea}. This contains a wide range of multivariate time-series classification problems from various fields with significant differences in time-series length, number of classes, and amount of training data. The full collection contains 30 datasets, however due to algorithm run-time constraints we have considered a subset of these ensuring they still contain significant variation. The statistics of these dataset are given in Table \ref{} of the Appendix.

The results are given in Table \ref{tab:uea_comparison_results}. We see improved performance from the old shapelet method in ? of ? cases. Whats more, we see there are datasets for which the optimal accuracy is achieved with a learnt metric and the logsignature discrepancy. These results, whilst not extensive, show the feasibility of improving performance by choosing a discrepancy tailored to the structure of the data as opposed to a simple euclidean distance metric.
\begin{table}[ht]
    \centering
    \caption{}
    \label{tab:uea_comparison_results}
    \begin{tabular}{lccc}
\toprule
{} & \multicolumn{3}{c}{\textbf{Discrepancy}} \\
\textbf{Dataset} &          $L^2$ &   Logsignature &                Classical \\
\midrule
BasicMotions    &    90.8\% $\pm$ 1.4\% &  80.8\% $\pm$ 3.8\% &  \textbf{96.7\% $\pm$ 5.8\%} \\
ERing           &    \textbf{82.6\% $\pm$ 6.3\%} &  43.3\% $\pm$ 2.9\% &  67.2\% $\pm$ 11.8\% \\
Epilepsy        &    \textbf{88.4\% $\pm$ 3.0\%} &  \textbf{88.6\% $\pm$ 0.8\%} &  72.9\% $\pm$ 5.4\% \\
Handwriting     &    10.3\% $\pm$ 2.6\% &  \textbf{11.8\% $\pm$ 1.2\%} &  6.5\% $\pm$ 3.7\% \\
JapaneseVowels  &    \textbf{97.2\% $\pm$ 1.1\%} &  53.9\% $\pm$ 3.0\% &  91.5\% $\pm$ 4.1\% \\
LSST            &    \textbf{36.1\% $\pm$ 0.2\%} &  35.7\% $\pm$ 0.4\% &  33.5\% $\pm$ 0.5\% \\
Libras          &    \textbf{67.0\% $\pm$ 9.4\%} &  \textbf{67.8\% $\pm$ 5.5\%} &  62.2\% $\pm$ 2.4\% \\
PenDigits       &    \textbf{97.3\% $\pm$ 0.1\%} &  96.7\% $\pm$ 0.7\% &  \textbf{97.5\% $\pm$ 0.6\%} \\
RacketSports    &    \textbf{79.6\% $\pm$ 0.7\%} &  61.2\% $\pm$ 9.2\% &  \textbf{79.6\% $\pm$ 2.4\%} \\
\midrule
Wins &            7 &                 3 &    3 \\
\bottomrule
\end{tabular}

\end{table}


\subsubsection{Algorithm Interpretability}
Here we explore the effect of the inclusion of the regularisation term from Equation \cite{eq:interpretable_reg} on shapelet interpretability. Recall that the term was chosen to ensure the resulting shapelets are `close', in the sense of the chosen discrepancy, to some subsample of the training data. To examine this we consider the PenDigits dataset where participants were asked to write down a number from 0-9 and the goal is to classify the intended digit. In Figure \ref{fig:pendigits} we plot, for each digit (0 to 9 in order from left to right), the learnt shapelet that corresponds to the largest coefficient from the logistic-regression for that class. For the old method \ref{fig:old_shapelets}, it is in general not clear what aspect of the digit the shapelet is representing. In some cases the closest digit to the shapelet is not even of the class for which it is thought to best discriminate. Furthermore, the 0 and 1 classes along with the 5 and 8 classes share their `top' shapelet. In contrast, it is abundantly clear what aspect of the digit is being captured by the top shapelets produced from our generalised method \ref{fig:new_shapelets}. Some particular points of interest include shape of the 5 and 6 shapelet being very similar, but the distinction being found in the shapelets chirality (along with its thickness) as the bottom half of a 5 is written clockwise and the 6 anticlockwise. The shapelet for the 7 digit is interesting in its simplicity with it not being immediately clear what makes it so discriminative. By considering those digits with strokes in the top left hand region, we can see that the 8 and 9 have different chiralities to the 7, and the strokes for 1, 2 and 3 are generally written curving upwards to being with wheras 7 is usually flat. A similar such case can be made to explain why the shapelet for the 2 is found to be highly discriminative.

We chose PenDigits here because it is easy to understand visually what the shapelet represents. In general this is not the case, but provided we can make sense of isolated subintervals of the time-series, then the generated results will be interpretable in this context.
\begin{figure}[ht]
    \begin{subfigure}[b]{\linewidth}
        \centering
        \includegraphics[width=\linewidth]{images/pendigits/old_shapelets_noborder.png}
        \caption{Old shapelets.}
        \label{fig:old_shapelets}
    \end{subfigure}
    \vspace{0.1}
    \begin{subfigure}[b]{\linewidth}
        \centering
        \includegraphics[width=\linewidth]{images/pendigits/new_shapelets_noborder.png}
        \caption{New shapelets.}
        \label{fig:new_shapelets}
    \end{subfigure}
    \caption{The learnt shapelet that corresponds to the largest coefficient in the logistic regression layer for the digits 0 to 9 in order from left to right (blue) and the `closest' path (the path with smallest discrepancy) from the training data (orange dashed).}
    \label{fig:pendigits}
\end{figure}


\subsubsection{Missing Data and Length Ablation}
We now demonstrate both the ability of the proposed framework to handle partially observed data, as well as show the effectiveness of having learnable shapelet lengths.
\begin{table}[ht]
    \caption{}
    \label{tab:uea_noise}
    \centering
    \begin{tabular}{lccc}
\toprule
& {} & \multicolumn{2}{c}{\textbf{Lengths selected by}} \\  \cmidrule{3-4}
\textbf{Dataset} & \parbox{20mm}{\centering\textbf{Dropped data}} & \parbox{20mm}{\centering Differentiable optimisation} & \parbox{20mm}{\centering Hyperparameter searching} \\
\midrule
\multirow{3}{*}{JapaneseVowels} & 10\% & \textbf{93.2\% $\pm$ 2.1\%} & \textbf{93.1\% $\pm$ 0.9\%} \\
                                & 30\% & \textbf{91.2\% $\pm$ 4.1\%} & \textbf{91.4\% $\pm$ 2.8\%} \\
                                & 50\% & \textbf{93.5\% $\pm$ 1.1\%} & 92.0\% $\pm$ 1.1\% \\
\multirow{3}{*}{Libras}         & 10\% & 57.4\% $\pm$ 4.2\% & \textbf{59.3\% $\pm$ 1.6\%} \\
                                & 30\% & \textbf{81.2\% $\pm$ 7.6\%} & 63.9\% $\pm$ 8.2\% \\
                                & 50\% & \textbf{62.5\% $\pm$ 14.8\%} & \textbf{65.3\% $\pm$ 8.6\%} \\
\multirow{3}{*}{LSST}           & 10\% & 40.2\% $\pm$ 3.5\% & \textbf{44.0\% $\pm$ 1.0\%} \\
                                & 30\% & \textbf{38.1\% $\pm$ 0.3\%} & \textbf{40.2\% $\pm$ 5.6\%} \\
                                & 50\% & 41.5\% $\pm$ 2.7\% & \textbf{44.4\% $\pm$ 0.5\%} \\
\midrule
Wins                            &    & 6 & 7 \\
\bottomrule
\end{tabular}



\end{table}


\subsection{Speech Commands}
Finally we examine the performance on the speech commands dataset \ref{warden2018speech} which includes a selection of one-second audio files with each representing a single spoken word. The aim is to build a model to detect the word that has been spoken. We chose this dataset as it is significantly larger than any in the UEA archive so as to demonstrate that the method is not restricted to these (relatively) small datasets. We do reiterate that the computational cost of shapelet methods is high in comparison to other more traditional deep-learning approaches and so can make datasets such as this prohibitive, in particular, it is why we have left the logsig-3 discrepancy out of our analysis in this instance as it takes significantly longer to train compared with the L2 methods.
\begin{table}[ht]
    \caption{Classification accuracy for old shapelets and new shapelets on the Speech Commands dataset.}
    \label{tab:speech_commands}
    \centering
    \begin{tabular}{cc}
\toprule
\multicolumn{2}{c}{Discrepancy} \\
Old &  L2 \\
\midrule
- & - \\
- & - \\
\bottomrule
\end{tabular}

\end{table}

\subsubsection{Interpretability of Speech Commands}
Pray for me.

